
% Thesis Abstract -----------------------------------------------------


%\begin{abstractslong}    %uncommenting this line, gives a different abstract heading
\begin{abstracts}        %this creates the heading for the abstract page

En esta tesis se entrenaron Redes Neuronales para la caracterización de rayos cósmicos utilizando detectores RICH (Ring Imaging Cherenkov). Los rayos cósmicos, partículas que permean el universo, pueden proporcionar información valiosa sobre la anturaleza de los antinúcleos detectados por el experimento AMS-02 proporcionado un entendimiento sobre la asimetría materia-antimateria. El objetivo principal es desarrollar y aplicar IA para contrastarla con algoritmos de verosimilitud -El método del estado del arte en RICH-  la precisión y eficiencia en la identificación y análisis de estos rayos.

Las técnicas de inteligencia artificial implementadas en este estudio demuestran una mejora significativa en la caracterización de los rayos cósmicos comparadas con los métodos tradicionales. Los resultados refuerzan la noción que la inteligencia artificial puede auxiliar significativamente la detección y análisis de partículas en futuros experimentos de física de partículas y astrofísica. Este trabajo plantea  un modelo eficiente y rápido, sugiriendo la continuación de esta línea de investigación para perfeccionar y ampliar sus aplicaciones.



\end{abstracts}
%\end{abstractlongs}


% ----------------------------------------------------------------------